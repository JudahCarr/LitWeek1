% Options for packages loaded elsewhere
\PassOptionsToPackage{unicode}{hyperref}
\PassOptionsToPackage{hyphens}{url}
%
\documentclass[
]{article}
\usepackage{amsmath,amssymb}
\usepackage{iftex}
\ifPDFTeX
  \usepackage[T1]{fontenc}
  \usepackage[utf8]{inputenc}
  \usepackage{textcomp} % provide euro and other symbols
\else % if luatex or xetex
  \usepackage{unicode-math} % this also loads fontspec
  \defaultfontfeatures{Scale=MatchLowercase}
  \defaultfontfeatures[\rmfamily]{Ligatures=TeX,Scale=1}
\fi
\usepackage{lmodern}
\ifPDFTeX\else
  % xetex/luatex font selection
\fi
% Use upquote if available, for straight quotes in verbatim environments
\IfFileExists{upquote.sty}{\usepackage{upquote}}{}
\IfFileExists{microtype.sty}{% use microtype if available
  \usepackage[]{microtype}
  \UseMicrotypeSet[protrusion]{basicmath} % disable protrusion for tt fonts
}{}
\makeatletter
\@ifundefined{KOMAClassName}{% if non-KOMA class
  \IfFileExists{parskip.sty}{%
    \usepackage{parskip}
  }{% else
    \setlength{\parindent}{0pt}
    \setlength{\parskip}{6pt plus 2pt minus 1pt}}
}{% if KOMA class
  \KOMAoptions{parskip=half}}
\makeatother
\usepackage{xcolor}
\usepackage[margin=1in]{geometry}
\usepackage{color}
\usepackage{fancyvrb}
\newcommand{\VerbBar}{|}
\newcommand{\VERB}{\Verb[commandchars=\\\{\}]}
\DefineVerbatimEnvironment{Highlighting}{Verbatim}{commandchars=\\\{\}}
% Add ',fontsize=\small' for more characters per line
\usepackage{framed}
\definecolor{shadecolor}{RGB}{248,248,248}
\newenvironment{Shaded}{\begin{snugshade}}{\end{snugshade}}
\newcommand{\AlertTok}[1]{\textcolor[rgb]{0.94,0.16,0.16}{#1}}
\newcommand{\AnnotationTok}[1]{\textcolor[rgb]{0.56,0.35,0.01}{\textbf{\textit{#1}}}}
\newcommand{\AttributeTok}[1]{\textcolor[rgb]{0.13,0.29,0.53}{#1}}
\newcommand{\BaseNTok}[1]{\textcolor[rgb]{0.00,0.00,0.81}{#1}}
\newcommand{\BuiltInTok}[1]{#1}
\newcommand{\CharTok}[1]{\textcolor[rgb]{0.31,0.60,0.02}{#1}}
\newcommand{\CommentTok}[1]{\textcolor[rgb]{0.56,0.35,0.01}{\textit{#1}}}
\newcommand{\CommentVarTok}[1]{\textcolor[rgb]{0.56,0.35,0.01}{\textbf{\textit{#1}}}}
\newcommand{\ConstantTok}[1]{\textcolor[rgb]{0.56,0.35,0.01}{#1}}
\newcommand{\ControlFlowTok}[1]{\textcolor[rgb]{0.13,0.29,0.53}{\textbf{#1}}}
\newcommand{\DataTypeTok}[1]{\textcolor[rgb]{0.13,0.29,0.53}{#1}}
\newcommand{\DecValTok}[1]{\textcolor[rgb]{0.00,0.00,0.81}{#1}}
\newcommand{\DocumentationTok}[1]{\textcolor[rgb]{0.56,0.35,0.01}{\textbf{\textit{#1}}}}
\newcommand{\ErrorTok}[1]{\textcolor[rgb]{0.64,0.00,0.00}{\textbf{#1}}}
\newcommand{\ExtensionTok}[1]{#1}
\newcommand{\FloatTok}[1]{\textcolor[rgb]{0.00,0.00,0.81}{#1}}
\newcommand{\FunctionTok}[1]{\textcolor[rgb]{0.13,0.29,0.53}{\textbf{#1}}}
\newcommand{\ImportTok}[1]{#1}
\newcommand{\InformationTok}[1]{\textcolor[rgb]{0.56,0.35,0.01}{\textbf{\textit{#1}}}}
\newcommand{\KeywordTok}[1]{\textcolor[rgb]{0.13,0.29,0.53}{\textbf{#1}}}
\newcommand{\NormalTok}[1]{#1}
\newcommand{\OperatorTok}[1]{\textcolor[rgb]{0.81,0.36,0.00}{\textbf{#1}}}
\newcommand{\OtherTok}[1]{\textcolor[rgb]{0.56,0.35,0.01}{#1}}
\newcommand{\PreprocessorTok}[1]{\textcolor[rgb]{0.56,0.35,0.01}{\textit{#1}}}
\newcommand{\RegionMarkerTok}[1]{#1}
\newcommand{\SpecialCharTok}[1]{\textcolor[rgb]{0.81,0.36,0.00}{\textbf{#1}}}
\newcommand{\SpecialStringTok}[1]{\textcolor[rgb]{0.31,0.60,0.02}{#1}}
\newcommand{\StringTok}[1]{\textcolor[rgb]{0.31,0.60,0.02}{#1}}
\newcommand{\VariableTok}[1]{\textcolor[rgb]{0.00,0.00,0.00}{#1}}
\newcommand{\VerbatimStringTok}[1]{\textcolor[rgb]{0.31,0.60,0.02}{#1}}
\newcommand{\WarningTok}[1]{\textcolor[rgb]{0.56,0.35,0.01}{\textbf{\textit{#1}}}}
\usepackage{graphicx}
\makeatletter
\def\maxwidth{\ifdim\Gin@nat@width>\linewidth\linewidth\else\Gin@nat@width\fi}
\def\maxheight{\ifdim\Gin@nat@height>\textheight\textheight\else\Gin@nat@height\fi}
\makeatother
% Scale images if necessary, so that they will not overflow the page
% margins by default, and it is still possible to overwrite the defaults
% using explicit options in \includegraphics[width, height, ...]{}
\setkeys{Gin}{width=\maxwidth,height=\maxheight,keepaspectratio}
% Set default figure placement to htbp
\makeatletter
\def\fps@figure{htbp}
\makeatother
\setlength{\emergencystretch}{3em} % prevent overfull lines
\providecommand{\tightlist}{%
  \setlength{\itemsep}{0pt}\setlength{\parskip}{0pt}}
\setcounter{secnumdepth}{-\maxdimen} % remove section numbering
\ifLuaTeX
  \usepackage{selnolig}  % disable illegal ligatures
\fi
\usepackage{bookmark}
\IfFileExists{xurl.sty}{\usepackage{xurl}}{} % add URL line breaks if available
\urlstyle{same}
\hypersetup{
  pdftitle={343.01: Class Markdown Key},
  pdfauthor={Oropendola},
  hidelinks,
  pdfcreator={LaTeX via pandoc}}

\title{343.01: Class Markdown Key}
\author{Oropendola}
\date{2024-08-20}

\begin{document}
\maketitle

\subsection{1: RMarkdown and R Code}\label{rmarkdown-and-r-code}

In an RMarkdown Document, you can write in prose, as you would in any
text editor. When you hit the ``Knit'' button, it will generate an
easy-to-read html, pdf, or docx file. We will use these to integrate
note-taking, class code, and assignments throughout the semester.

When editing text, we'll start with the basics: \emph{italics} are
triggered with one set of asterisks, and \textbf{bold} with two sets.
For simplicity's sake, we will \textbf{not} indent paragraphs, and
instead use double line breaks.

When we want to run code in R, we need to start an ``R Chunk'', we will
set them up like the text below. In today's class, we will have
\texttt{echo\ =\ TRUE} or \texttt{=\ T}, so that we can see the code in
the html output.

\begin{Shaded}
\begin{Highlighting}[]
\CommentTok{\# Our first code}
\end{Highlighting}
\end{Shaded}

At its most basic, R is a fancy calculator, capable of performing math
quickly and reproducibly. We will often use annotations (which always
start with a `\#') to help us remember what a bit of code does, or to
flag something to come back to later

\begin{Shaded}
\begin{Highlighting}[]
\CommentTok{\# Do spaces matter?}

\CommentTok{\# multiplication}

\CommentTok{\# division}

\CommentTok{\# parentheses to protect order of operations}

\CommentTok{\# exponents}
\end{Highlighting}
\end{Shaded}

R is object-oriented, which means it is structured around data objects
that have names, to which we perform functions or from which we make new
products, like plots!

An example of a named data object is a vector: a series of values with a
name. We can create and ask questions above related vectors. In R, we
start with a name and then use ``\textless-'' to indicate what is in
that name.

\begin{Shaded}
\begin{Highlighting}[]
\CommentTok{\# high temps from four days}

\CommentTok{\# low temps from four days}

\CommentTok{\# preview the data in the console by running the name}

\CommentTok{\# check to see if an object is a vector}

\CommentTok{\# create a new vector of temperature changes}

\CommentTok{\# preview the new vector in the console}
\end{Highlighting}
\end{Shaded}

\subsection{2: data.frames, functions, and
arguments}\label{data.frames-functions-and-arguments}

Data.frames are a common table structure used in R. Each row is an
observation, each column is a variable. Each column should have the same
data type: there should only be numeric data in a temperature column,
only categorical or factor data in a ``species'' column, etc.

We will usually create data.frames by assigning a name, and then
importing data from csv files using the \textbf{function}
\texttt{read.csv()}. A \textbf{function} is a pre-coded program that
performs an operation. We specify how that function works by adding
\textbf{arguments}.

For example, R does not always know to treat text from a csv file as
factors. On most operating systems, \texttt{read.csv()} will leave text
as character or strings, and not sort a column into category levels.

\begin{Shaded}
\begin{Highlighting}[]
\CommentTok{\# read in the weather data without arguments}
\CommentTok{\# my.name \textless{}{-} function("filename.csv")}

\CommentTok{\# use the function "summary()" to find out what is in the data.frame}
\end{Highlighting}
\end{Shaded}

You can see that R is smart enough to identify numbers, and summarizes
them with ranges and averages. It will notes missing values as
\texttt{NA} without issue.

However, it is ``blind'' to the \textbf{character} values in
station\_id, location, and date. It considers each one as a unique
STRING of text.

Let's modify our \texttt{read.csv()} function with an \textbf{argument}
called \texttt{stringsAsFactors} to correct this.

\begin{Shaded}
\begin{Highlighting}[]
\CommentTok{\# start the code the same way}

\CommentTok{\# use the function "summary()" to see if it worked!}
\end{Highlighting}
\end{Shaded}

R now ``knows'' the names of different factors - like the letter for
cardinal directions, or the name of the airport the measurement was
taken from.

\texttt{summary()} is just one exploratory function. We can also use
\texttt{str()} to find a more concise vertical summary, or
\texttt{head()} to see the first few rows.

\subsection{3: Navigating data.frames with {[},{]}, \$, and more
functions}\label{navigating-data.frames-with-and-more-functions}

The BIGGEST benefit of R is reproducibility: anyone can use the same
code on the same data to get the same results, with (usually) no fear of
over-writing a dataset or having an invisible formula in a cell, like
you do in Excel.

However, it isn't as easy to \emph{see} your data in R, especially for
large datasets. As such, we need to be comfortable navigating our
data.frames with syntax and functions. While we learn to do this, we
will also discover that R is quite picky about cAse sENsitvity and
sp3lling.

\begin{Shaded}
\begin{Highlighting}[]
\CommentTok{\# Open a tab with a (big) preview of the data.frame.}
\CommentTok{\#View(weather) \# this works, but we DON\textquotesingle{}T want it in the final markdown}
\CommentTok{\#view(weather) \# this does not work {-} command needs to be capital V}
\CommentTok{\#View(Weather) \# this does not work {-} data.frame starts with a lowercase w}
\end{Highlighting}
\end{Shaded}

We will often want to know things about specific columns, rows, or
specific cells within a data.frame. If we know the ``position'' of
something within our data.frame, we can use the
\texttt{name{[}row\#,column\#{]}} to view or extract it.

\begin{Shaded}
\begin{Highlighting}[]
\CommentTok{\# how many rows?}

\CommentTok{\# how many columns?}

\CommentTok{\# the first row of the data.frame}

\CommentTok{\# the first row and the second column}

\CommentTok{\# The first three rows and the first three columns}

\CommentTok{\# The cell from th 80th row and the seventh column}
\end{Highlighting}
\end{Shaded}

This is useful, but tedious. After all, these are not just floating
values -- these are variables! We'll almost always be asking questions
about the \emph{columns} of our dataset, which we know have names.

\begin{Shaded}
\begin{Highlighting}[]
\CommentTok{\# get a list of column names}
\end{Highlighting}
\end{Shaded}

To make life easy for us, R will give us a drop-down menu of column
names any time we type the data.frame's name followed by a dollar sign -
which means we can run mathematical and statistical \textbf{functions}
on them!

\begin{Shaded}
\begin{Highlighting}[]
\CommentTok{\# get a summary of the location column}

\CommentTok{\# find the mean of the maximum temperature}

\CommentTok{\# find the median of minimum temperature}

\CommentTok{\# find the standard deviation of precipitation}

\CommentTok{\# find the range of wind speed}
\end{Highlighting}
\end{Shaded}

\subsection{4: Basic plots}\label{basic-plots}

R comes with a simple plotting package. When we use it, we will usually
use \texttt{plot()} function to make xy scatterplots or boxplots, and
\texttt{hist()} for histograms. The syntax among these plots is shared;
the arguments specify what data.frame\$columns go on which axis, and
what the labels and main title should be.

\begin{Shaded}
\begin{Highlighting}[]
\CommentTok{\# R\textquotesingle{}s base plotting package}
\DocumentationTok{\#\# You can make a lot of graphs by typing plot()}
\DocumentationTok{\#\# plot(x = data.frame$column,}
\DocumentationTok{\#\#      y = data.frame$column,}
\DocumentationTok{\#\#      type = …,}
\DocumentationTok{\#\#      labs = …)}
\CommentTok{\# A: Ugly scatterplot}

\DocumentationTok{\#\# simplified with defaults…}

\CommentTok{\# B: Prettier scatterplot:}

\DocumentationTok{\#\# simplified with defaults…}
\end{Highlighting}
\end{Shaded}

If one variable is categorical and the other is numeric, R should be
smart enough to switch from a scatterplot to a boxplot.

\begin{Shaded}
\begin{Highlighting}[]
\DocumentationTok{\#\# station\_id is a categorical}
\end{Highlighting}
\end{Shaded}

Sometimes, we just want to know the distribution of a single numeric
variable. In these cases, histograms are our best friends! We only input
the column of interest on the x-axis, because R will tabulate the
frequency or number of observations on the y-axis for us.

\subsection{5: Dealing with dates \&
times}\label{dealing-with-dates-times}

Time data is incredibly annoying\ldots humans write it differently
depending on culture and context, leading to all manner of confusion
during analysis. Computers prefer for dates to be read in as YYYY-MM-DD,
but anyone who has ever opened Excel knows that such formatting goes out
the window the instant a file is saved or updated.

Fortunately, there is a package in R called \texttt{lubridate} which
specializes in wrangling date information into readable formats, and
then extracting relevant parts of time data. To use it, we must first
install it, and then require it into the R session.

\begin{Shaded}
\begin{Highlighting}[]
\CommentTok{\# bring the package in}
\end{Highlighting}
\end{Shaded}

Okay, here's where object orientation really starts to matter: we need
to \textbf{overwrite} the original \texttt{weather\$date} column using
the lubridate \texttt{ymd()} function to extract date information from
the \texttt{\$date} column.

\begin{Shaded}
\begin{Highlighting}[]
\CommentTok{\# what is the current data type of weather$date?}

\CommentTok{\# overwrite it as a date using ymd()}

\CommentTok{\# did it work?}
\end{Highlighting}
\end{Shaded}

We can use \texttt{lubridate} to create new columns that hold just a
portion of the date column, such as the year or month. This is confusing
in its own way, because sometimes the name of the column is the same as
the name of the function.

\begin{Shaded}
\begin{Highlighting}[]
\CommentTok{\# make a year column}

\CommentTok{\# make a month column}
\end{Highlighting}
\end{Shaded}

Time data is also annoying because in some cases, we should treat it as
numeric, and other times we should treat it as a factor. For example,
January is month number 1, February month 2, but they don't make much
sense when we consider the graphical relationship between month and
minimum temperatures.

\begin{Shaded}
\begin{Highlighting}[]
\CommentTok{\# plot with month on x, tmin on y}
\end{Highlighting}
\end{Shaded}

In this context, month makes far more sense as an ordinal category.
Fortunately, we can use the \texttt{as.X()} family of functions to
re-assign the data.type on the fly.

\begin{Shaded}
\begin{Highlighting}[]
\CommentTok{\# summary of month as is}

\CommentTok{\# summary of a factor version of month}

\CommentTok{\# replot with factor version of month}
\end{Highlighting}
\end{Shaded}

\subsection{6: Nice graphs with ggplot2}\label{nice-graphs-with-ggplot2}

R's base plots are fine, but professional plots are often made with the
R package \texttt{ggplot2}. Like lubridate, it needs to be installed and
then required for it to work in an R session.

\begin{Shaded}
\begin{Highlighting}[]
\FunctionTok{require}\NormalTok{(ggplot2) }\CommentTok{\#or tidyverse}
\end{Highlighting}
\end{Shaded}

\begin{verbatim}
## Loading required package: ggplot2
\end{verbatim}

After that, we have to get familiar with the ggplot syntax\ldots{}

\begin{Shaded}
\begin{Highlighting}[]
\CommentTok{\# ggplot is the function {-} not ggplot2}
\CommentTok{\# ggplots follow a distinct syntax or logic from R\textquotesingle{}s}
\DocumentationTok{\#\# base plotting package:}
\DocumentationTok{\#\#\#  ggplot(data.frame,}
\DocumentationTok{\#\#\#        aes(x= column, }
\DocumentationTok{\#\#\#            y = column)) +}
\DocumentationTok{\#\#\#     geometry()…  }

\DocumentationTok{\#\# A: Simple ggplot2 scatterplot}


\DocumentationTok{\#\# B: Complicated ggplot2 scatterplot}


\DocumentationTok{\#\# C: A simplified complicated plot that}
\DocumentationTok{\#\#\# splits up the two station\_ids into facets}


\DocumentationTok{\#\# D: Pretty histogram of wind speeds}
\end{Highlighting}
\end{Shaded}

We can even make some pretty cool graphs, in which we can look at the
frequency of wind direction at our two different airports.

\paragraph{R resources}\label{r-resources}

\end{document}
